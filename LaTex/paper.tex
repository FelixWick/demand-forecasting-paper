\documentclass[BCOR=1mm, DIV=calc,10pt,
twoside=true,
twocolumn,
headings=normal]{scrartcl}
%\KOMAoptions{DIV=calc}% recalculate the page layout with a calculated DIV value

% Recommended, but optional, packages for figures and better typesetting:
\usepackage{microtype}
\usepackage{graphicx}
\usepackage{subfigure}
\usepackage{booktabs} % for professional tables

\usepackage{amsmath}
\usepackage{bm}
\usepackage{url}

% Attempt to make hyperref and algorithmic work together better:
\newcommand{\theHalgorithm}{\arabic{algorithm}}

\usepackage{cite}
\usepackage{amsmath,amssymb,amsfonts}
\usepackage{algorithmic}
\usepackage{graphicx}
\usepackage{textcomp}
\usepackage{xcolor}

\usepackage{authblk}


\newcommand{\fig}{Fig. }
\newcommand{\eqn}{Eqn. }
\newcommand{\tab}{Tab. }
\newcommand{\wrt}{w.r.t. }
\newcommand{\etal}{ {\em et al. }} 


\begin{document}


%\title{Prediction and Evaluation of individual Probability Density Functions with Machine Learning}
\title{Demand Forecasting of individual Probability Density Functions with Machine Learning}

\author[1]{Felix Wick \thanks{felix.wick@blueyonder.com}}
\author[2]{Ulrich Kerzel \thanks{u.kerzel@iubh-fernstudium.de}}
\author[1]{Trapti Singhal \thanks{trapti.singhal@blueyonder.com}}
\author[1]{Martin Hahn \thanks{martin.hahn@blueyonder.com}}
\affil[1]{\small Blue Yonder GmbH (Karlsruhe, Germany)}
\affil[2]{\small IUBH Internationale Hochschule (Erfurt, Germany)}


\date{}

\maketitle

\begin{abstract}
... abstract ...
\end{abstract}

{Keywords: \textbf{machine learning, demand forecasting}}


\section{Introduction}

Demand forecasting is one of the main challenges for retailers and at the core of business
operations. Due to its stochastic nature, demand is difficult to forecast as it depends on
many influencing factors and the realized demand can be interpreted as a random variable
that is described by an appropriate probability density function (PDF). In order to make
operational decisions, an optimal point estimator has to be defined that  can be used to
derive ordering decisions used in the replenishment process of the retailer. Demand
estimation is further complicated by the fact that retailers typically only observed
realized sales and not the actual demand, in case the demand exceeds the current stock
level, the data become censored. This decision process is complicated by a range of
factors: Even in the case of accurate  demand forecasts, the decision maker has to balance
conflicting metrics to reach an optimal decision: Ordering to few items may result in
stockout situations resulting in unrealized demand and unsatisfied customers. Ordering too
many items results in excess inventory which increases transport and storage costs and, in
the case of perishable goods, excessive waste as spoilt items need to be disposed of at
additional cost in addition to  the environmental impact. This situation is particularly
noticeable in the so called `'ultra-fresh'' category which includes items such as bakery
products, ready-meals, fresh diary products or certain meat products such as ground meat.
These items typically have a shelf-life from less than a business day to a few business
days at most with a continuous spectrum  depending on the exact item. In many situations,
additional constraints have to be considered to reach an optimal ordering decision:
Delivery cycles of items may vary depending on the type of item and the wholesaler or
manufacturer from which  they are procured. Retailers also operate at a given service
level to guarantee that a certain level of demand can be fulfilled. The exact service
level typically depends on the overall business policy of the retailer and may also depend
on individual products, ranging from ``never-out-of-stock'' items to a service level
exceeding e.g. 90\%.

The availability of Big Data allows capturing, storing and processing a vast amount of
data associated with demand such as historic sales records,  information about promotional
events or advertisements, pricing information, local weather at retail locations, seasonal
information  as well as a wide range of further variables. Modern machine learning
algorithms can then be used to predict the per-item demand distribution, corrected for
censored data from which an optimal point estimator can be derived to be used in the
subsequent ordering decision. It is important to note that demand as a random variable is
not identically and independently distributed (i.i.d.): While the probability distribution
describing the demand can be attributed to a given family or parametrization, the exact
parameters vary: Seasonal effects, finite life cycles of products and the introduction of
new products influence the demand distribution, as well as the local weather at the retail
location or the retail location itself in terms of size, assortment range, customer
diversity and other factors. The retailers themselves also actively influence demand by
using advertisements to highlight products, offering rebates or discounts for specific
products as well as pursuing an active pricing strategy. This means that while we can
generally assume that demand follows a specific type of probabilty distribution, its
parameters are unique to the  instance for which an estimate is required. For example, the
probability distribution governing the demand of a particular item is specific to the
item, date and retail location for which the forecast is made and depends on a wide range
of further influencing factors.

The remainder of the paper is organized as follows:
We first review the relevant literature and existing work in sec. \ref{sec:LitRev}.
We then describe our method to predict individual negative binomial PDFs by means of a
parametric approach including two distinct machine learning models for mean of variance in
sec. \ref{sec:pdfEstimation}.
After that we describe methods for the qualitative and quantitative evaluation of PDF
predictions in sec. \ref{sec:pdfEvaluation}.
Finally, we present a demand forecasting example to show an application of our methods in
sec. \ref{sec:example}.


\section{Literature Review}
\label{sec:LitRev}

Inventory management offers a rich theory and the extensive body of research can be
broadly grouped into the following two categories where the inventory control problem is
either based on some knowledge of the underlying demand  distribution or an integrated
approach that seeks to map directly from the available data (historic sales records and 
further variables) to the ordering decision. The latter approach is taken e.g. in
\cite{beutel2012safety,ban2019big,bertsimas2020predictive} and aims to avoid estimating
the underlying probability distribution. Although this approach seems preferable since it
avoids determining the full demand distribution and results directly in the desired
operational decision (order quantity), it faces several drawbacks. First of all, the full
probability distribution for the demand of an specific item at a given sales location and
business day includes all available information including the uncertainty of the modelled
demand. This can be used to simulate the performance on a per-item level and e.g. optimize
the impact on business strategy decisions on conflicting metrics such as stock out- and
waste-rate. Additionally, having the full demand probability distribution available allows
for an accute assessment of the forecast quality at all quantiles of the distribution,
including the often extensive tails of the distribution, as well as the analysis of
long-term  effect as the demand predictions including all future planned advertisements or
price-changes can be included into long-term forecasts. Furthermore, deriving the ordering
decision  directly couples the demand process with the complex delivery cycles and
constraints, keeping the steps separate allows greater operational flexibility and reduces
the impact of changing manufacturers or wholesalers, as well as  allowing a quick response
to changes in the delivery schedule wihtout having to re-compute the implicit demand
underlying the ordering decision. Finally, long-term demand forecasts may be shared with
other business units or external vendors  and wholesalers to ease their planning for the
production and supply-chain processes upstream of the retailer.

In contrast, more traditional inventory control systems rely on the knowledge of the
demand distribution in one form or another.  see e.g. \cite{silver1998} for an overview.
In $(s,S)$ type inventory control systems \cite{Scarf1958}, inventory levels are monitored
at regular intervals and orders are dispatched once the inventory level reaches a minimal
value $s$. In case of linear holding and shortage costs, such policies are optimal
\cite{Scarf1959}, although perishable goods pose more challenges, see e.g.
\cite{Nahmias1973,Nahmias1975,nahmias1978}. Additionally, service level constraints can be
included in these kind of inventory control systems \cite{minner2010periodic}. Perishable
goods are well described by the `'newsvendor-problem'' \cite{Edgeworth} where in the
simplest case all stock perishes at the end of the selling period (e.g. a business day). 
For a detailed review of the newsvendor problem see e.g. \cite{Khouja1999537}. Assuming
linear underage and overage costs $b,h >0$, the optimal quantile
$q_{\mathrm{opt}} = {b}/{(b+h})$ of a known demand distribution $f(D)$ can be calculated
exactly. The main objective in any of these approaches is to determine the underlying
demand distribution. The simplest approach is to just use the observed sales events and
forecast these as a time-series  (see e.g. \cite{alwan2016}) or via sample average
approximation (SAA). (see e.g. \cite{shapiro2014} for an overview over SAA). However,
these approaches do not make use of any data apart from the sales record themselves,
although we know that many variables such as price, advertisements, etc. are highly
correlated with demand. Additionally, it is critical that we indeed reconstruct a demand
distribution, hence a simple point-estimator as provided by the most common statistical
techniques and machine-learning approaches will not suffice. Additionally, demand is not
identically and independently distributed (i.i.d.) but depends not only on external
factors such as season, weather, product life-cycle, but is also actively changed by the
retailer by setting a specific price, offering rebates or running advertisements.
Additionally, the demand implicitly depend on the location of the retail outlet as well as
the specifics of that location such as product  assortment influencing the choice of
possible replacement articles and many more. These complications are the main reason we
cannot treat the replenishment process as $n$ independent newsvendor-type problems.
Instead, we need to determine the full demand distribution from data, conditional on the
relevant variables such as date, location and item, taking all auxiliary data such as
article charactersitics, pricing, advertisements, retail location details, etc. into
account. This can be done in several ways: First, we can use a neural network
\cite{Feindt2006190} to learn the distribution from data and return a full distribution
per item, store and day from which the relevant quantile can be estimated. Similarly,
using quantile regression \cite{koenker2001} this approach can be implemented in different
frameworks. Alternatively, one can assume a given demand model and fit the model
parameters instead of reconstructing the complete distribution. This approach is
computationally favourable, as fewer parameters need to be estimated compared to the case
of the full distribution. Empirically, one can determine the best fitting distribution
from data \cite{adan1995}. However, given the stochastic nature of the demand, such an
empirically determined distribution is not expected to be stable and prone to sudden
changes. Instead, the choice of the demand distribution should be motivated by theoretic
considerations. The discrete demand is typically modelled as a negative binomial
distribution (NBD), also known as Gamma-Poisson distribution
\cite{Ehrenberg1959,Ehrenberg1967,Ehrenberg1972,Chatfield1973,Schmittlein_1985}. This
distribution arises if the Poisson parameter $\mu$ is a random variable itself that
follows a Gamma distribution. The NBD has two parameters, $\mu$ and $ \sigma^2 > \mu$ and
is over-dispersed compared to the Poisson distribution for which $\mu = \sigma^2$.

Hence, for each ordering decision, the model parameters $\mu$ and $\sigma$ need to be
determined for each item at each sales location and ordering time, depending on all
auxiliary data describing article details, retail location and influencing factors such as
pricing and advertisement information.

\subsection*{Summary of Contributions}
What we are going to show.


\section{Negative Binomial PDF Estimation}
\label{sec:pdfEstimation}

To predict an individual PDF by means of a parametric approach, one has to rely on a model
assumption about the underlying distribution of the random variable to be predicted. In
the following, we describe such a parametric approach with the model assumption of a
negative binomial probability distribution. For a PDF prediction under a negative binomial
model assumption, there is the need for the estimation of two parameters, for example its
mean and its variance.
In any case, as there is a strong correlation between mean and variance, it is beneficial to include the corresponding
predicted mean from the mean model as feature in the variance model.

This can be done using two independent models one,  to estimate the mean and the other
for the variance. At least in principle, any method can be used. However, as argued above,
in case of demand forecasting, each prediction is highly specific to the circumstances
in which it is used (such as product ID, day and store location) and may
depend on  a multitude of describing variables (features).
Machine learning algorithms are ideally suited for this task and in the
following we will use the ``Cyclic Boosting'' algorithm \cite{Wick2019}.
The major benefits of this algorithm are that it is not only extremly performant in practical 
applications of demand forecasting at large scale, but also fully explainable. Traditional machine
learning algorithms are typically ``black box'' approaches where the individual decision cannot be 
explained. Cyclic Boosting on the other hand allows to follow how each individual prediction was made.
The main idea of this algorithm is the following: When used for regression,
the predictions $\hat{y_i}$ of the  target variable $Y \in [0,\infty)$ can be calculated in the following
way:
\begin{equation} \label{eqn:cb}
\hat{y}_i = \mu \cdot \prod \limits_{j=1}^p f^k_j \quad \text{with}\; k=\{ x_{j,i} \in b^k_j\}
\end{equation}
The parameters $f^k_j$ are the model parameters that are determined from 
features $j$. Each feature is discretisized appropriately into $k$ bins to reflect the 
specific behaviour of the feature. The global mean $\mu$ is determined from all observed
target values $y$ observed in the data. The factors  $f^k_j$ are determined
iteratively until convergence is reached and regularization techniques are applied
to avoid overtraining and improve the generalization ability of the algorithm.
The relative strength of the factors  $f^k_j$ are then directly interpretable in
relation how important a specific feature is for each individual prediction where deviations
from  $f^k_j=1$ indicate high importance.

%One can achieve this with two subsequent, independent machine learning models, the first
%to estimate the mean and the second to estimate the variance. The features may or may not
%differ between the mean and variance estimation models. 


The assigned mean and variance predictions can then be used to generate individual
probability density functions according to a functional negative binomial distribution
assumption for each sample.

%While any machine learning algorithm outputting a conditional mean as prediction may be
%used to get an approximate estimate for the mean of an underlying negative binomial
%distribution, it is preferable to choose an algorithm using a negative binomial assumption
%for its inner regularization methods. An example for such a machine learning algorithm is
%Cyclic Boosting in multiplicative regression mode .
%In the upcoming section, we describe a method to predict the variance of a negative
%binomial PDF along with its mean individually for each sample by means of Cyclic Boosting.

\subsection{Variance Estimation by Cyclic Boosting in Negative Binomial Width Mode}

Cyclic Boosting in a modified form of its multiplicative regression mode, namely the
negative binomial width mode presented in the following, can be used for the second machine
learning model to predict the negative binomial variance associated with the mean
predicted by the first model. 

For regression, the parametrization of the negative binomial mass function
can be specified as \cite{hilbe2011negative}:

\begin{equation} \label{eqn:nbinom}
\mathrm{NB}(y; \mu, r) = \frac{\Gamma(r + y)}{y! \cdot \Gamma(r)} \cdot \left(\frac{r}{r + \mu}\right)^r \cdot \left(\frac{\mu}{r + \mu}\right)^y,
\end{equation}
with mean $\mu$ and dispersion parameter $r$. The target variable $y$ takes the values $y = 0, 1, 2, ...$.

The variance estimation is achieved by minimizing the loss function defined in \eqn
\eqref{eqn:loss_likelihood}, expressed as negative log-likelihood function of a negative
binomial distribution, with respect to $r_i$ over all input samples $i$, where the mean
values $\hat{\mu_i}$ are fixed to the corresponding predictions from the first machine
learning model outputting the conditional mean.

\begin{equation} \label{eqn:loss_likelihood}
L(r) = -\mathcal{L}(r) = -\ln \sum_i \mathrm{NB}(y_i; \hat{\mu_i}, r_i)
\end{equation}

As stated in \eqn \eqref{eqn:r} the values $r_i$ are defined by the Cyclic Boosting
model parameters $f^k_j$ for each feature $j$ and bin $k$. For any concrete observation
$i$, the index $k$ of the bin is determined by the observation of $x_{j,i}$ and the
subsequent look-up into which bin this observation falls. The dispersion parameter $r$ is
bound to the interval $[1, \infty]$.

\textcolor{red}{Hm, this part is a bit quick. We should add a sentence to the NB loss 
and how this is motivated and then how this is connected to \eqn \eqref{eqn:r} and how
\eqn \eqref {eqn:cb} from CB comes into play.}
\begin{equation} \label{eqn:r}
r_i = 1 + \prod \limits_{j=1}^p f^k_j \quad \text{with}\; k=\{ x_{j,i} \in b^k_j\}
\end{equation}

\textcolor{red}{Hm, this part is not quite clear - need to add some details what is meant}
As described in \cite{Wick2019}, the parameter estimation in Cyclic Boosting is an
iterative method corresponding to a cyclic coordinate descent, processing one feature with
all its bins at a time until convergence. Unlike in the original multiplicative regression
mode of Cyclic Boosting, the minimization of the loss function in \eqn
\eqref{eqn:loss_likelihood} cannot be solved analytically and requires a numerical method,
for example a random search. All other advantages of Cyclic Boosting presented in
\cite{cyclicboosting}, like for example individual explainability of predictions, remain
valid for its negative binomial width mode.

Finally, the variance $\hat{\sigma}^2_i$ can be calculated from the dispersion parameter
$\hat{r_i}$ by means of \eqn \eqref{eqn:variance_r}. And so, together with the estimated
mean parameter $\hat{\mu_i}$ from the previous mean estimation model, all parameters of
the negative binomial distribution NB for the prediction of an individual observation $i$
are given.

\begin{equation} \label{eqn:variance_r}
\sigma^2 = \mu + \frac{\mu^2}{r}
\end{equation}


\section{Evaluation of PDF Predictions}
\label{sec:pdfEvaluation}

Statistical or machine learning methods that predict full individual probability functions typically lack
quantitative, or at least qualitative, evaluation of the PDF model output to assess its
correctness. In the case of an estimation of the determining parameters of an assumed
functional form for the PDF, assessing the correctness of the PDF model output refers to
the evaluation of the accuracy of the prediction of the different determining parameters,
e.g. mean and variance of a negative binomial distribution, and the validation of the
model assumptions, i.e. the choice of the underlying PDF, by comparing the probability
density estimations to observed data. However, the two methods presented in this section,
namely cumulative distribution function (CDF) histograms and inverse quantile plots, are
neither restricted to the evaluation of parametric PDF estimations nor to a specific
functional form, like a negative binomial, but can be used generically.

\subsection{Qualitative Evaluation of PDF Predictions}

\subsubsection{CDF Histogram}
\label{sec:cdf_histo}

\textcolor{red}{need to explain intuition a bit more, a diagram would be helpful.}
The idea here is that the individual PDF predictions are first transformed into individual
CDF predictions and the CDF values, also known as quantiles, of the corresponding truth
values are then filled to a histogram.

\textcolor{red}{need to explain a bit more what we see and why we expect a uniform distribution.}
Figure \eqref{fig:cdf_histos} illustrates five different CDF histograms. If the PDF
prediction is correct, a flat uniform distribution should occur for the CDF histogram,
like illustrated by the final far-right column of shaded circles.

In the example of a parametric negative binomial PDF estimation of section
\eqref{sec:pdfEstimation}, this would indicate that both the mean and the variance
estimations as well as the choice of a negative binomial distribution as underlying PDF
were correct. Conversely, the first two plots of \fig \ref{fig:cdf_histos} indicate a
bias in the variance estimation and the third and fourth plot indicate a bias in the mean
estimation.
\textcolor{red}{Need to explain why we see that. Would be good to show the respective distributions
(or at least one example) of wher it doesn't fit.}

\begin{figure}
\begin{center}
\includegraphics[width=8cm]{../figures/cdf_histos}
\caption{\label{fig:cdf_histos} ...}
\end{center}
\end{figure}

\subsubsection{Inverse Quantile Plot}

\textcolor{red}{We need to explain a lot more here. In particular, outside HEP the profile
plots are not known so we need to explain that first and then explain why we want to use
them here.}
The idea here is that the individual PDF predictions are first transformed into individual
CDF predictions and the CDF values, also known as quantiles, of the corresponding truth
values are then compared (in the sense of higher or lower) to expected quantile values and
accordingly filled to a collection of profile plots, each of which representing one
expected quantile value.

\textcolor{red}{since no reader of this paper will have seen a profile plot before or
know what that is, we need to guide the audience step-by-step to what they should see and why it's
correct and what each column means, how we come to the conclusion, etc.}
\fig \ref{fig:invquant_profiles} illustrates five different collections of inverse
quantile profile plots (each collection comparing to 7 expected quantile values), each
with only one bin for the sake of visualization, for separate sets of exemplary
probability density estimation and observed data combinations. Each shown line represents
the percentage of probability density estimation and observed data combinations for which
the observed data point should be above and below the quantile of the predicted PDF
indicated by that line (\textcolor{red}{median?}for example, the 0.50 line indicates that 50 percent of all
probability density estimation and observed data combinations in a given set of data
should fall above the line, and 50 percent should fall below the line), respectively. The
observation of the number of samples, indicated with shaded circles, that do in fact fall
above and below a particular line, then allows the evaluation of the accuracy of
probability density estimations. For a set of accurate PDF predictions, one expects a
uniform shape, like illustrated by the final far-right column of shaded circles.

\textcolor{red}{need to explain how one can see this}
In the example of a parametric negative binomial PDF estimation of section
\ref{sec:pdfEstimation}, this would indicate that both the mean and the variance
estimations as well as the choice of a negative binomial distribution as underlying PDF
were correct. Conversely, the leftmost two columns of shaded circles illustrate a case for
which the variance estimation has a bias and the center and center-right columns of shaded
circles illustrate a case for which the mean estimation has a bias.

\begin{figure}
\begin{center}
\includegraphics[width=8cm]{../figures/invquant_profiles}
\caption{\label{fig:invquant_profiles} ...}
\end{center}
\end{figure}

The two advantages of this method, as compared to the simpler method presented in section
\ref{sec:cdf_histo}, are that an inverse quantile plot supports the qualitative evaluation
of the predicted individual PDFs not only globally but (1) for different specified
quantiles (potentially hinting to deviations in e.g. the tails of the distributions) and
(2) in dependence of arbitrary variables of the data set (potentially hinting to
deviations in e.g. specific categories of a feature). Two examples for this can be found
in figures \ref{fig:invquant_mean} and \ref{fig:invquant_pg} in the next section.

\subsection{Quantitative Evaluation of PDF Predictions}

The methods described so far allow a detailed qualitative evaluation of PDF predictions.
However, in order to also quantify the quality of the PDF predictions, a measure of the
deviation of the PDF predictions from the optimal outcome given the observed target data
is needed. To achieve this, we compare the CDF histogram of the predicted PDFs with the
uniform distribution and define a prediction accuracy interval between 0 and 1.

Several different methods can be used to compute the deviance between two probability
distributions. Good choices are the \textbf{Wasserstein metric} \cite{olkin1982}, a
distance function defined between two probability distributions on a given metric space
(also known as earth mover’s distance), and the \textbf{Kullback-Leibler divergence}
\cite{kullback1951}, a measure of how one probability distribution diverges from a second
expected probability distribution. The first has the advantage that, unlike the second, it
takes an underlying metric space into account, meaning that it depends on the distance of
potential deviations.


\section{Example: Demand Forecasting}
\label{sec:example}

... predict an individual demand volume for different product-location-date combinations ...
... example showing full prediction and evaluation chain ...

... plot showing individual PDF example ...

%\begin{figure}
%\begin{center}
%\includegraphics[width=8cm]{../figures/pdf_example}
%\caption{\label{fig:pdf_example} ...}
%\end{center}
%\end{figure}

... plot showing cdf histo ...

%\begin{figure}
%\begin{center}
%\includegraphics[width=8cm]{../figures/cdf_demand}
%\caption{\label{fig:cdf_demand} ...}
%\end{center}
%\end{figure}

... plot showing invquant profile for mean prediction on X-axis ...

%\begin{figure}
%\begin{center}
%\includegraphics[width=8cm]{../figures/invquant_mean}
%\caption{\label{fig:invquant_mean} ...}
%\end{center}
%\end{figure}

... plot showing invquant profile for product groups on X-axis ...

%\begin{figure}
%\begin{center}
%\includegraphics[width=8cm]{../figures/invquant_pg}
%\caption{\label{fig:invquant_pg} ...}
%\end{center}
%\end{figure}



...


\bibliography{paper}
\bibliographystyle{ieeetr}

%%
%% Appendices
%%
\appendix

\section{Profile Histograms}
In many cases, scatter plots are used to study the behaviour of two distributions or sets of data points visually. 
However, even for moderate amount of data, this approach becomes quickly difficult.
To illustrate this, a sample of $(x,y)$ data points was obtained in the following way:
The distribution of $x$ values was obtained by generating 5,000 samples of Gaussian
distributed random numbers $X \sim {\cal N}(0.0,2.0)$ and the $y$ values are obtained via $Y \sim X +  {\cal N}(2.0,1.5)$.
\fig \ref{fig:scatter} shows the marginal distributions for $x$ and $y$ as well as a scatter plot of $x$ vs. $y$.

\begin{figure}
\begin{center}
\includegraphics[scale=0.5]{../figures/marginal} \includegraphics[scale=0.5]{../figures/scatter} 
\caption{\label{fig:scatter} Marginal distribution and scatter plot of variables $X$ and $Y$.}
\end{center}
\end{figure} 

Although the simple linear correlation between $X$ and $Y$ is apparent in the scatter plot, finer details are not
visible and it is easy to imagine that a more complex relationship is difficult to discern.
Profile histograms are specifically designed to address this shortcomming. Intuitively, profile
histograms are a one-dimensional representation of the two-dimensional scatter plot and are obtained 
in the following way: The variable on the $x$ axis is discretizised into a suitable range of bins.
The exact choice of binning depends on the problem at hand. One can for example choose equidistant bins
in the range of the $x$ axis or non-equidistant bins such that each bin contains the same number of observations.
Then within each bin of the variable $X$, the a location and dispersion metric is calculated for the variable $Y$.
This means that the bin-borders on the $X$ axis are used as constraints on the variable $Y$ and with these
conditions applied, for example the sample mean of the selected $y$ values as well as the standard deviation
are calculated.
These location and dispersion metrics in each bin of $X$ are used to illustrate the behaviour of the variable $Y$
as the values of the variable $X$ change from bin to bin.
The resulting profile histogram is shown in \fig \ref{fig:profile}. This one-dimensional representation
allows to understand even a complex relationship between two variables visually. Note
that due to few data points at the edges of the distributions the profile histogram is expected
to show visual artifacts in the corresponding regions.
 \begin{figure}
\begin{center}
\includegraphics[scale=0.5]{../figures/profile}
\caption{\label{fig:profile} Profile histogram of variables $X$ and $Y$.}
\end{center}
\end{figure} 





\end{document}
